\documentclass[12pt]{article}

\usepackage[utf8]{inputenc}
\usepackage[T1]{fontenc}
\usepackage[francais]{babel}
\usepackage{color}
\usepackage{graphicx}
\usepackage{url}
\usepackage{hyperref}
\usepackage[a4paper, top=1.5cm, bottom=1.5cm, left=1.5cm, right=1.5cm]{geometry}

\usepackage{amsthm}
\usepackage{amsmath}
\usepackage{amssymb}
\usepackage{mathrsfs}
\usepackage{mathrsfs}
\usepackage{dsfont}
\usepackage{fourier-orns}
\usepackage{xcolor} % \colorbox{yellow}{texte} surlignera alors le texte avec la couleur jaune

\usepackage{dsfont} % pour l'indicatrice : \mathds{1} produira un 1 comme l'indicatrice
\usepackage{float}


\newtheorem{Lem}{Lemme}[section]
\newtheorem{Theo}{Th\'{e}or\`{e}me}[section]
\newtheorem{Cor}{Corollaire}[section]
\newtheorem{Prop}{Proposition}[section]
\newtheorem{Def}{D\'{e}finition}[section]
\newtheorem{Not}{Notation}[section]

\newcommand{\Rmq}{\noindent\textbf{Remarque : }}
\newcommand{\Csq}{\noindent\textbf{Cons\'{e}quence : }}
\newcommand{\Rpl}{\noindent\textbf{Rappel : }}

\newcommand{\R}{\mathbb{R}}
\newcommand{\ssi}{si et seulement si }

% Pour la proba :
%\renewcommand{\P}{\mathscr{P}}
%\newcommand{\courbe}[1]{\mathscr{#1}}

% Pour les espaces de matrices
\newcommand{\Mn}[1]{\mathscr{M}_n(\mathbb{#1})}

% Pour la classe d'une fonction :
\newcommand{\C}[1]{\mathcal{C}^{#1}}

% pour faciliter l'\'{e}criture des fonctions
\newcommand{\fonc}[5]{\begin{array}{r r c l}
#1 : & #2 &\rightarrow & #3\\
& #4 & \mapsto & #5
\end{array}}

% d droit des d\'{e}riv\'{e}es
\newcommand{\dd}{\mathrm{d}}

% pour faciliter l'\'{e}criture de la somme en dehors des \[\]
\newcommand{\somme}[2]{\underset{#1}{\overset{#2}{\sum}}}

\newcommand\independent{\protect\mathpalette{\protect\independenT}{\perp}}
\def\independenT#1#2{\mathrel{\rlap{$#1#2$}\mkern2mu{#1#2}}}
% variables ind\'{e}pendantes

\author{Laura \bsc{Greige} \and Thibault \bsc{Gigant}}
\date{\today}
\title{Cahier des Charges\\P-ANDROIDE}



\begin{document}

%
%\begin{figure}
%\includegraphics[width=5cm]{upmc-logotype.png}
%\end{figure}
%
%\begin{figure}
%\includegraphics[width=5cm]{LogoBrown.png}
%\end{figure}

\begin{titlepage}

\newcommand{\HRule}{\rule{\linewidth}{0.7mm}} % Defines a new command for the horizontal lines, change thickness here

\center % Center everything on the page
 
%----------------------------------------------------------------------------------------
%	HEADING SECTIONS
%----------------------------------------------------------------------------------------
%\begin{flushleft}
%\begin{minipage}{0.4\textwidth}
%%\begin{flushleft}
%\begin{center}
%\includegraphics[width=5cm]{upmc-logotype.png}
%
%\medskip
%\textsc{\large Universit\'{e}\\ Pierre-et-Marie-Curie}%\\[1.5cm] % Name of your university/college
%\end{center}
%%\end{flushleft}
%\end{minipage}
%\end{flushleft}
%\hfill


%\textsc{\Large Major Heading}\\[0.5cm] % Major heading such as course name
%\textsc{\large Minor Heading}\\[0.5cm] % Minor heading such as course title

%\bigskip
\hspace{3cm}
\vspace{2cm}

%----------------------------------------------------------------------------------------
%	TITLE SECTION
%----------------------------------------------------------------------------------------

\HRule \\[0.4cm]
{ \huge \bfseries Cahier des Charges\\[0.4cm]P-ANDROIDE}\\[0.4cm] % Title of your document
\HRule \\[2.5cm]
 
%----------------------------------------------------------------------------------------
%	AUTHOR SECTION
%----------------------------------------------------------------------------------------

\begin{minipage}{0.4\textwidth}
\begin{flushleft} \large
\emph{Auteurs :}\\
Laura \textsc{Greige}\\ % Your name
Thibault \textsc{Gigant}\\ % Your name
\end{flushleft}
\end{minipage}
~
\begin{minipage}{0.4\textwidth}
\begin{flushright} \large
\emph{Encadrants :} \\
Olivier \textsc{Spanjaard}\\ % Supervisor's Name
\end{flushright}
\end{minipage}\\[2cm]


%----------------------------------------------------------------------------------------
%	DATE SECTION
%----------------------------------------------------------------------------------------
\vspace{3cm}
{\large 2015 -- 2016}\\[2cm] % Date, change the \today to a set date if you want to be precise

%----------------------------------------------------------------------------------------
%	LOGO SECTION
%----------------------------------------------------------------------------------------

\vfill
\begin{center}
\includegraphics[width=5cm]{upmc-logotype.png}

\medskip
\textsc{\large Universit\'{e}\\ Pierre-et-Marie-Curie}%\\[1.5cm] % Name of your university/college
\end{center}
 
%----------------------------------------------------------------------------------------

\vfill % Fill the rest of the page with whitespace

\end{titlepage}

\section*{Introduction}

\section{Les deux approches : Branch \& Bound et S\'{e}riation}


\subsection{Branch \& Bound}

\subsection{S\'{e}riation}

\section{R\'{e}alisation}


\end{document}